\documentclass[12pt]{article}
\usepackage[a4paper,margin=1in,footskip=0.25in]{geometry} % set margins
\usepackage[portuguese]{babel}
\usepackage[utf8]{inputenc}
\usepackage[hidelinks]{hyperref} 
\usepackage{amsmath}
\usepackage{amssymb}
\usepackage{amsthm}
\usepackage{graphicx}    % needed for include graphics
\usepackage{subfigure}   % add subfigures
\usepackage{indentfirst}
\usepackage{float}       % needed for [H] figure placement option
\usepackage{setspace}    % needed for doublespacing
\usepackage{tikz}
\usepackage{algpseudocode}
\usepackage{listings}
\usepackage{xcolor}

\definecolor{mGreen}{rgb}{0,0.6,0}
\definecolor{mGray}{rgb}{0.5,0.5,0.5}
\definecolor{mPurple}{rgb}{0.58,0,0.82}
\definecolor{backgroundColour}{rgb}{0.95,0.95,0.92}

\lstdefinestyle{CStyle}{
    backgroundcolor=\color{backgroundColour},   
    commentstyle=\color{mGreen},
    keywordstyle=\color{magenta},
    numberstyle=\tiny\color{mGray},
    stringstyle=\color{mPurple},
    basicstyle=\footnotesize,
    breakatwhitespace=false,         
    breaklines=true,                 
    captionpos=b,                    
    keepspaces=true,                 
    numbers=left,                    
    numbersep=5pt,                  
    showspaces=false,                
    showstringspaces=false,
    showtabs=false,                  
    tabsize=2,
    language=C
}

% Macros
\renewcommand{\familydefault}{\sfdefault} % sans-serif
\newcommand{\lowtext}[1]{$_{\text{#1}}$}
\newcommand{\code}[1]{\texttt{#1}}

% Adds ./img/ to the path of figures
\graphicspath{{./img/}}

\title{Relatório EP2 - MAC0219}
\author{Bruno Sesso, Gustavo Estrela de Matos}

\begin{document}
% Espaçamento duplo 
\doublespacing
\begin{titlepage}
    \vfill
    \begin{center}
        \vspace{0.5\textheight}
        \noindent
        Instituto de Matemática e Estatística \\
        EP1 - MAC0219 \\
        \vfill
        \noindent
        {\Large Implementação de Algoritmos de Criptografia e Codificação
                Paralelos Usando CUDA} \\
        \bigskip
        \bigskip
        \begin{tabular}{ll}
            {\bf Professor:} & {Alfredo Goldman} \\
            {\bf Alunos:}    & {Bruno Sesso} \\
                             & {Gustavo Estrela de Matos} \\
        \end{tabular} \\
        \vspace{\fill}
       \bigskip
        São Paulo, \today \\
       \bigskip
    \end{center}
\end{titlepage}

\pagebreak
\tableofcontents
\pagebreak


%%%%%%%%%%%%%%%%%%%%% INTRODUÇÃO %%%%%%%%%%%%%%%%%%%%%%%%%%%%%%%%%%%%%%%
\newpage
\section{Introdução}
Este trabalho tem como objetivo a paralelização e análise de desempenho
de algoritmos de criptografia e codificação para serem rodados em GPUs.
O código desenvolvido utilizará a biblioteca \emph{Compute Unified 
Device Architecture} (CUDA), e portanto será compatível apenas com GPUs 
NVIDIA.

Pensando na arquitetura \emph{multiple instruction single data} (MISD), 
escolhemos três algoritmos em que os dados não possuem grandes 
dependência entre si, e portanto podem ser separados mais facilmente 
para serem processados em paralelo. Os três algoritmos escolhidos foram:
\begin{itemize}
    \item{\textbf{Base64}}: um algoritmo de codificação;
    \item{\textbf{Rot13}}: também um algoritmo de codificação;
    \item{\textbf{Vigenere}}: um algoritmo de cifração.
\end{itemize}
        
Para analisar a paralelização dos algoritmos a nossa principal métrica
foi o tempo de execução ao processar arquivos de texto. Os três 
algoritmos escolhidos não tem grandes dependências ao conteúdo lido, 
portanto escolhemos arbitrariamente uma versão em texto puro da bíblia
para medir tempos de execução. Para garantir significância estatística
utilizamos o programa \emph{perf}, que nos permite apresentar resultados
médios de rodadas do algoritmo.

% todo: falar onde a gente rodou as paradas. Qual as configs da placa?
O computador usado nos testes...

%%%%%%%%%%%%%%%%%%%%% ALGORITMOS %%%%%%%%%%%%%%%%%%%%%%%%%%%%%%%%%%%%%%%
% aqui a gente explica os algoritmos que a gente escolheu e mostra
% como a versão sequencial deles funciona
\newpage
\section{Algoritmos Escolhidos}
\subsection{Base64}
\subsection{Rot13}
\subsection{Vigenere}
% todo: explicar que mudou o algoritmo


%%%%%%%%%%%%%%%%%%%%% CUDA %%%%%%%%%%%%%%%%%%%%%%%%%%%%%%%%%%%%%%%%%%%%%
% aqui a gente mostra como paraleliza os algoritmos
\newpage
\section{Código Paralelo}
\subsection{Base64}
\subsection{Rot13}
\subsection{Vigenere}


%%%%%%%%%%%%%%%%%%%%% DISCUSSÃO DOS RESULTADOS %%%%%%%%%%%%%%%%%%%%%%%%%
\newpage
\section{Discussão dos Resultados}


%%%%%%%%%%%%%%%%%%%%% CONCLUSAO %%%%%%%%%%%%%%%%%%%%%%%%%%%%%%%%%%%%%%%%
\newpage
\section{Conclusão}

\end{document}
